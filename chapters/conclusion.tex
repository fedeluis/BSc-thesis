\chapter{Conclusioni}

In questo capitolo sono riportate le considerazioni riguardanti l'attività di stage.

\section{Raggiungimento degli obiettivi}

Lo stage si è svolto rispettando i tempi prefissati e tutti gli obiettivi obbligatori, desiderabili e facoltativi sono stati terminati seguendo il piano di lavoro redatto prima dell'inizio dello stage.
\begin{itemize}
    \item Obbligatori:
    \begin{itemize}
        \item Acquisizione delle competenze sulle tematiche dell'architettura a microservizi, del framework Spring e del broker RabbitMQ
        \item Capacità di raggiungere gli obiettivi richiesti in autonomia seguendo il cronoprogramma
        \item Portare a termine le implementazioni previste con una percentuale di superamento pari al 80\%
    \end{itemize}
    \item Desiderabili:
    \begin{itemize}
        \item Portare a termine le implementazioni previste con una percentuale di superamento pari al 100\%
    \end{itemize}
    \item Facoltativi:
    \begin{itemize}
        \item Dockerizzare le componenti su container
        \item Creare un sistema di log
    \end{itemize}
\end{itemize}

\section{Conoscenze acquisite}

Nella sfera delle conoscenze acquisite sono molto soddisfatto di essermi potuto avvicinare allo sviluppo software lato \textit{back-end}. Il progetto mi ha permesso di studiare a fondo il \textit{framework} Spring di cui avevo solamente sentito parlare e di capire cos'è e come funziona un broker di messaggi. Il periodo di tirocinio mi ha anche permesso di dare uno sguardo al mondo del lavoro inserendomi nelle dinamiche aziendali e permettendomi di collaborare con colleghi più esperti nel settore, i quali mi hanno dato utili consigli.
