\chapter{Challenginator}\label{challenginator}

L'analisi che segue è stata svolta da persone terze quando il progetto è stato avviato \cite{canovese}.\\
\textit{Challenginator} è una \textit{web application} che permette l'invio, la ricezione ed il tracciamento di sfide tra i membri di un team. Mediante la \textit{user interface} ogni utente può effettuare la registrazione e, dopo il processo di autenticazione tramite login, l'accesso alla propria area personale.
Da quest'ultima l'utente ha accesso a:
\begin{itemize}
    \item Home: pagina riepilogativa
    \item Dashboard: pagina con accesso rapido alle sfide passate, alla procedura di lancio di una nuova sfida ed elencazione di sfide in cui si è stati sfidati, si sfida o si fa da valutatore
    \item Preferenze di notifica: pagina dove l'utente decide quali tipologie di notifica ricevere
    \item Storico: pagina contenente lo storico delle sfide
    \item Nuova Challenge: pagina dove è possibile lanciare una nuova sfida
    \item Logout
\end{itemize}
Ogni utente può sfidare un collega attraverso l'apposito form di inserimento di una nuova sfida ed in modo automatico verrà assegnato un valutatore. Quest'ultimo è il soggetto, gerarchicamente a livello superiore, che decreterà la riuscita od il fallimento della sfida una volta completata da parte dello sfidato.

%%%%%%%%%%  ANALISI REQUISITI   %%%%%%%%%%
\section{Analisi dei requisiti}

Lo scopo del prodotto è stato riportato nella sezione precedente.
Durante l'intero flusso dell'applicazione gli attori del sistema sono:
\begin{itemize}
    \item \textbf{Utente non autenticato:} identifica un utente che non ha ancora effettuato l'accesso al rispettivo account
    \item \textbf{Utente autenticato:} identifica un utente che ha effettuato correttamente il login e può essere identificato a sua volta come:
    \begin{itemize}
        \item \textbf{Sfidante:} utente che ha inserito una sfida contro un utente terzo
        \item \textbf{Sfidato:} utente che ha ricevuto una sfida da un utente terzo
        \item \textbf{Valutatore:} utente che deve decretare la riuscita o meno di una sfida
    \end{itemize}
\end{itemize}

%%%%%%%%%%  CASI D'USO  %%%%%%%%%%
\subsection{Casi d'uso}

I casi d'uso individuati sono elencati di seguito.

\begin{table}[H]
    \centering
    \begin{tabular}{|p{3cm}p{11.15cm}|}
        \hline
        %\multicolumn{2}{|c|}{\textbf{UC-1: Login utente}} \\\hline
        \textbf{Attore primario} & utente non autenticato \\
        \textbf{Descrizione} & autenticazione utente \\
        \textbf{Precondizioni} & l’utente non si è ancora autenticato nell’applicazione \\
        \textbf{Input} & l’utente inserisce ed invia i dati per il login \\
        \textbf{Postcondizioni} & il cliente è autenticato \\\hline
    \end{tabular}
    \caption{UC-1: Login utente}
\end{table}
\begin{table}[H]
    \centering
    \begin{tabular}{|p{3cm}p{11.15cm}|}
        \hline
        %\multicolumn{2}{|c|}{\textbf{UC-1: Login utente}} \\\hline
        \textbf{Attore primario} & utente non autenticato \\
        \textbf{Descrizione} & registrazione nuovo utente \\
        \textbf{Precondizioni} & l’utente non si è ancora autenticato nell’applicazione \\
        \textbf{Input} & l’utente inserisce ed invia i dati per la creazione del proprio account \\
        \textbf{Postcondizioni} & il cliente è autenticato \\\hline
    \end{tabular}
    \caption{UC-2: Registrazione utente}
\end{table}
\begin{table}[H]
    \centering
    \begin{tabular}{|p{3cm}p{11.15cm}|}
        \hline
        %\multicolumn{2}{|c|}{\textbf{UC-1: Login utente}} \\\hline
        \textbf{Attore primario} & utente autenticato (sfidante, sfidato, valutatore) \\
        \textbf{Descrizione} & l’utente vuole visualizzare le sfide in cui è coinvolto come sfidato, sfidante o valutatore \\
        \textbf{Precondizioni} & l’utente ha effettuato il login \\
        \textbf{Input} & l’utente clicca sulla voce di menù dashboard, se non la sta già visualizzando a seguito del login \\
        \textbf{Postcondizioni} & l’utente visualizza la \textit{dashboard} contenente tutte le sfide in cui è coinvolto a vario titolo \\\hline
    \end{tabular}
    \caption{UC-3: Visualizzazione lista sfide (\textit{dashboard})}
\end{table}
\begin{table}[H]
    \centering
    \begin{tabular}{|p{3cm}p{11.15cm}|}
        \hline
        %\multicolumn{2}{|c|}{\textbf{UC-1: Login utente}} \\\hline
        \textbf{Attore primario} & utente autenticato - sfidante \\
        \textbf{Descrizione} & l’utente che ha lanciato la sfida vuole cancellarla \\
        \textbf{Precondizioni} & l’utente sta visualizzando la dashboard riepilogativa ed è l’utente sfidante \\
        \textbf{Input} & l’utente clicca sul pulsante per cancellare la sfida \\
        \textbf{Postcondizioni} & l’utente ha cancellato la sfida \\\hline
    \end{tabular}
    \caption{UC-3.1: Cancellazione challenge}
\end{table}
\begin{table}[H]
    \centering
    \begin{tabular}{|p{3cm}p{11.15cm}|}
        \hline
        %\multicolumn{2}{|c|}{\textbf{UC-1: Login utente}} \\\hline
        \textbf{Attore primario} & utente autenticato - sfidato \\
        \textbf{Descrizione} & l’utente vuole accettare una sfida che gli viene proposta \\
        \textbf{Precondizioni} & l’utente sta visualizzando la \textit{dashboard} riepilogativa, e una sfida in cui è identificato come sfidato è in attesa \\
        \textbf{Input} & l’utente clicca sul pulsante per accettare la sfida \\
        \textbf{Postcondizioni} & l’utente ha accettato la sfida \\\hline
    \end{tabular}
    \caption{UC-3.2: Accettazione challenge}
\end{table}
\begin{table}[H]
    \centering
    \begin{tabular}{|p{3cm}p{11.15cm}|}
        \hline
        %\multicolumn{2}{|c|}{\textbf{UC-1: Login utente}} \\\hline
        \textbf{Attore primario} & utente autenticato - sfidato \\
        \textbf{Descrizione} & l’utente vuole rifiutare una sfida che gli viene proposta \\
        \textbf{Precondizioni} & l’utente sta visualizzando la \textit{dashboard} riepilogativa, e una sfida in cui è identificato come sfidato è in attesa \\
        \textbf{Input} & l’utente clicca sul pulsante per rifiutare la sfida \\
        \textbf{Postcondizioni} & l’utente ha rifiutato la sfida \\\hline
    \end{tabular}
    \caption{UC-3.3: Rifiuto challenge}
\end{table}
\begin{table}[H]
    \centering
    \begin{tabular}{|p{3cm}p{11.15cm}|}
        \hline
        %\multicolumn{2}{|c|}{\textbf{UC-1: Login utente}} \\\hline
        \textbf{Attore primario} & utente autenticato - sfidato \\
        \textbf{Descrizione} & l’utente vuole dichiarare la sfida in cui è sfidato come completata \\
        \textbf{Precondizioni} & l’utente sta visualizzando la \textit{dashboard} riepilogativa, e vuole dichirare la sfida (in cui risulta sfidato) come completata \\
        \textbf{Input} & l’utente clicca sul pulsante per completare la sfida \\
        \textbf{Postcondizioni} & l’utente ha modificato lo stato della sfida \\\hline
    \end{tabular}
    \caption{UC-3.4: Dichiarare la sfida come completata}
\end{table}
\begin{table}[H]
    \centering
    \begin{tabular}{|p{3cm}p{11.15cm}|}
        \hline
        %\multicolumn{2}{|c|}{\textbf{UC-1: Login utente}} \\\hline
        \textbf{Attore primario} & utente autenticato - sfidato \\
        \textbf{Descrizione} & l’utente vuole arrendersi in una sfida \\
        \textbf{Precondizioni} & l’utente sta visualizzando la \textit{dashboard} riepilogativa, e vuole arrendersi in una sfida \\
        \textbf{Input} & l’utente clicca sul pulsante per arrendersi \\
        \textbf{Postcondizioni} & l’utente ha modificato lo stato della sfida \\\hline
    \end{tabular}
    \caption{UC-3.5: Arrendersi in una sfida}
\end{table}
\begin{table}[H]
    \centering
    \begin{tabular}{|p{3cm}p{11.15cm}|}
        \hline
        %\multicolumn{2}{|c|}{\textbf{UC-1: Login utente}} \\\hline
        \textbf{Attore primario} & utente autenticato \\
        \textbf{Descrizione} & l’utente vuole visualizzare il dettaglio di una sfida \\
        \textbf{Precondizioni} & l’utente sta visualizzando la \textit{dashboard} riepilogativa \\
        \textbf{Input} & l’utente clicca sul pulsante per visualizzare i dettagli della sfida, quali inizio e fine, tempo rimanente, la descrizione completa... \\
        \textbf{Postcondizioni} & l’utente visualizza la pagina di dettaglio della sfida \\\hline
    \end{tabular}
    \caption{UC-4: Visualizzazione dettaglio sfida}
\end{table}
\begin{table}[H]
    \centering
    \begin{tabular}{|p{3cm}p{11.15cm}|}
        \hline
        %\multicolumn{2}{|c|}{\textbf{UC-1: Login utente}} \\\hline
        \textbf{Attore primario} & utente autenticato \\
        \textbf{Descrizione} & l’utente vuole inserire una nuova sfida \\
        \textbf{Precondizioni} & l’utente clicca il pulsante di creazione nuova sfida \\
        \textbf{Input} & l’utente inserisce i dati richiesti per lanciare una sfida \\
        \textbf{Postcondizioni} & l’utente ha lanciato una sfida \\\hline
    \end{tabular}
    \caption{UC-5: Inserimento nuova sfida}
\end{table}
\begin{table}[H]
    \centering
    \begin{tabular}{|p{3cm}p{11.15cm}|}
        \hline
        %\multicolumn{2}{|c|}{\textbf{UC-1: Login utente}} \\\hline
        \textbf{Attore primario} & utente autenticato \\
        \textbf{Descrizione} & l’utente vuole visualizzare il dettaglio di una sfida \\
        \textbf{Precondizioni} & l’utente sta visualizzando la \textit{dashboard} riepilogativa \\
        \textbf{Input} & l’utente clicca il pulsante per visualizzare il dettaglio della sfida \\
        \textbf{Postcondizioni} & l’utente visualizza la pagina di dettaglio della sfida \\\hline
    \end{tabular}
    \caption{UC-6: Visualizzazione storico sfide}
\end{table}
\begin{table}[H]
    \centering
    \begin{tabular}{|p{3cm}p{11.15cm}|}
        \hline
        %\multicolumn{2}{|c|}{\textbf{UC-1: Login utente}} \\\hline
        \textbf{Attore primario} & utente autenticato \\
        \textbf{Descrizione} & l’utente vuole decidere quali notifiche ricevere \\
        \textbf{Precondizioni} & l’utente seleziona le \textit{checkbox} \\
        \textbf{Input} & l’utente clicca il pulsante di selezione preferenze di notifica, come in figura \ref{fig:prefnot} \\
        \textbf{Postcondizioni} & l’utente salva le sue preferenze di notifica \\\hline
    \end{tabular}
    \caption{UC-7: Scelta preferenze di notifica}
\end{table}

%%%%%%%%%% TRACCIAMENTO REQUISITI   %%%%%%%%%%
\subsection{Tracciamento dei requisiti}
Per elencare i requisiti individuati correlati ai casi d'uso utilizzo la seguente codifica:
\begin{center}
    \texttt{R[Importanza][Tipologia]-[Codice]}
\end{center}
\begin{itemize}
    \item \textbf{Importanza:} indica l'importanza di tale requisito attraverso i valori
    \begin{itemize}
        \item \textbf{1} requisito obbligatorio
        \item \textbf{2} requisito desiderabile
        \item \textbf{3} requisito opzionale
    \end{itemize}
    \item \textbf{Tipologia}
    \begin{itemize}
        \item \textbf{V} requisito di vincolo
        \item \textbf{F} requisito funzionale
        \item \textbf{Q} requisito di qualità
    \end{itemize}
    \item \textbf{Codice:} identificatore univoco in forma gerarchica padre/figlio
\end{itemize}
\begin{center}
    \textbf{[CodiceBase](.[CodiceSottoCaso])*}
\end{center}
Il CodiceBase identifica il caso d'uso generico. Il CodiceSottoCaso (opzionale) identifica i sottocasi.

\begin{longtable}[H]{ll}
    \caption{Tabella tracciamento dei requisiti funzionali} \\
    \textbf{Requisito} & \textbf{Descrizione} \\\hline
    R1F-1 & L’utente deve poter fare il login per accedere al sito \\\hline
    R1F-1.2 & Il sistema deve fare il display dell’errore di autenticazione \\\hline
    R1F-2 & L’utente non autenticato deve potersi registrare \\\hline
    R1F-2.1 & Il sistema deve fare il display dell’errore di registrazione \\\hline
    R1F-3 & L’utente deve poter visualizzare la dashboard \\\hline
    R1F-3.1 & L’utente deve poter visualizzare le sfide in cui è coinvolto come sfidato \\\hline
    R1F-3.1 & L’utente deve poter visualizzare le sfide in cui è coinvolto come sfidante \\\hline
    R1F-3.1 & L’utente deve poter visualizzare le sfide in cui è coinvolto come valutatore \\\hline
    R1F-3.2 & L’utente deve poter accettare una sfida che gli viene proposta \\\hline
    R1F-3.3 & L’utente deve poter cancellare una sfida lanciata \\\hline
    R1F-3.4 & L’utente deve poter segnare come completata una sfida \\\hline
    R1F-3.5 & L’utente deve poter arrendersi in una sfida \\\hline
    R1F-3.6 & L’utente valutatore deve poter valutare con succeso una sfida \\\hline
    R1F-3.7 & L’utente valutatore deve poter valutare il fallimento di una sfida \\\hline
    R1F-4 & L’utente deve poter inserire una nuova sfida \\\hline
    R1F-4.1 & L’utente deve poter selezionare chi sfidare \\\hline
    R1F-4.2 & L’utente deve poter specificare il titolo della sfida \\\hline
    R1F-4.3 & L’utente deve poter specificare il dettaglio della sfida \\\hline
    R1F-4.4 & L’utente deve poter inserire il termine per completare la sfida \\\hline
    R1F-5 & L’utente deve poter visualizzare lo storico delle sfide \\\hline
    R1F-6 & L’utente deve poter visualizzare il dettaglio di una sfida \\\hline
    R1F-6.1 & L’utente deve poter visualizzare il dettaglio cronologico della sfida \\\hline
    R1F-7 & L'utente deve poter decidere quali notifiche ricevere \\\hline
\end{longtable}

\begin{longtable}[H]{ll}
    \caption{Tabella del tracciamento dei requisiti di vincolo} \\
    \textbf{Requisito} & \textbf{Descrizione} \\\hline
    R1V-1 & Il backend deve essere realizzato tramite il framework Spring \\\hline
    R1V-2 & Il frontend deve essere realizzato tramite il framework Angular \\\hline
    R1V-3 & Il backend deve adottare un’architettura a microservizi \\\hline
    R1V-3 & La persistenza dei dati deve essere garantita con un database PostgreSQL \\\hline
    R1V-4 & La GUI deve essere realizzata mediante l’utilizzo della libreria \textit{Bootstrap} \\\hline
\end{longtable}

%%%%%%%%%%  PROBLEMATICHE   %%%%%%%%%%
\subsection{Problematiche riscontrate}
Dall'analisi preventiva dei rischi sono state individuate alcune possibili problematiche a cui si potrà andare incontro:
\begin{enumerate}
    \item \textbf{Conoscenza delle tecnologie}
    \begin{itemize}
        \item \textbf{Descrizione:} possibile rallentamento nell'attività di sviluppo a causa dell'utilizzo delle nuove tecnologie acquisite nella fasi di studio individuale \ref{teoria}
        \item \textbf{Soluzione:} autoverifica periodica delle conoscenze acquisite, testandole con piccoli progetti
    \end{itemize}
    \item \textbf{Inclusione di nuovi requisiti}
    \begin{itemize}
        \item \textbf{Descrizione:} l'utilizzo di un metodo di sviluppo \textit{agile} potrebbe complicare l'aggiunta di nuovi requisiti da soddisfare in corso d'opera
        \item \textbf{Soluzione:} coinvolgere il committente, in questo caso il tutor, in una ciclica analisi dei requisiti, così da intervenire solamente per piccoli aggiustamenti
    \end{itemize}
\end{enumerate}

%%%%%%%%%%  PROGETTAZIONE   %%%%%%%%%%
\section{Progettazione}

L'architettura della componente \textit{backend} del progetto \textit{Challenginator} è realizzata con i microservizi, ovvero servizi di piccole dimensioni che per poter comunicare tra loro utilizzano API ben definite.
La comunicazione con il lato \textit{frontend} avviene utilizzando chiamate all'API Gateway, microservizio che si interpone tra \textit{frontend} e \textit{backend}.
La comunicazione tra microservizi avviene mediante il broker \gls{rabbitmq}.\\
Il progetto è composto dai seguenti microservizi:
\begin{itemize}
    \item api-gateway
    \item challenge-service
    \item eurekaserver
    \item logger-service
    \item notification-service
    \item scheduler-service
    \item user-service
\end{itemize}
Il database che permette il salvataggio permanente dei dati è composto dalle seguenti tabelle:\\
Tabella \texttt{app\_user} per salvare i dati riguardanti gli utenti. Le password prima di esser salvate saranno crittografate.
\begin{table}[H]
    \centering
    \begin{tabular}{|p{4cm}|}
    \hline
    \textbf{app\_user}\\\hline
    \underline{id}\\
    app\_user\_role\\
    boss\_id\\
    email\\
    enabled\\
    locked\\
    name\\
    password\\
    score\\
    surname\\\hline
    \end{tabular}
    \caption{Tabella \texttt{app\_user} del database \texttt{challenginator}}
\end{table}

Tabella \texttt{log} per salvare i log dei vari microservizi.
\begin{table}[H]
    \centering
    \begin{tabular}{|p{4cm}|}
    \hline
    \textbf{log}\\\hline
    \underline{id}\\
    level\\
    log\_message\\
    service\_name\\
    time\\\hline
    \end{tabular}
    \caption{Tabella \texttt{log} del database \texttt{challenginator}}
\end{table}

Tabella \texttt{challenge} per salvare i dettagli delle singole sfide.
\begin{table}[H]
    \centering
    \begin{tabular}{|p{4cm}|}
    \hline
    \textbf{challenge}\\\hline
    \underline{id}\\
    challenged\\
    challenger\\
    deadline\\
    description\\
    evaluator\\
    result\\
    status\\
    timestamp\_acceptance\\
    timestamp\_creation\\
    title\\\hline
    \end{tabular}
    \caption{Tabella \texttt{challenge} del database \texttt{challenginator}}
\end{table}

Tabella \texttt{user\_preference} per salvare le preferenze di notifica di ogni utente. I campi di questa tabella (escluso \texttt{id}) sono di tipo \texttt{boolean}.
\begin{table}[H]
    \centering
    \begin{tabular}{|p{4cm}|}
    \hline
    \textbf{user\_preference}\\\hline
    \underline{userid}\\
    challenge\_accepted\\
    challenge\_completed\\
    challenge\_created\\
    challenge\_deleted\\
    challenge\_giveup\\
    challenge\_refused\\
    challenge\_terminated\\
    challenge\_updated\\\hline
    \end{tabular}
    \caption{Tabella \texttt{user\_preference} del database \texttt{challenginator}}
\end{table}