\chapter{Progetto personale}

Dopo una prima fase puramente teorica il tirocinio è proseguito con un progetto personale per poter testare le conoscenze apprese. Quest'ultimo doveva includere il \textit{framework} Spring, il \textit{broker} RabbitMQ e permettere il salvataggio persistente dei dati su database PostgreSQL. Il progetto \cite{projtutorial} scelto aveva il compito di gestire le ordinazioni provenienti da più ristoranti. La struttura che lo compone è riportata di seguito:
\begin{flushleft}
\dirtree{%
.1 springbootrabbitmqexample.
.2 config.
.3 MessagingConfig.java.
.2 consumer.
.3 User.java.
.2 dto.
.3 Order.java.
.3 OrderRepository.java.
.3 OrderStatus.java.
.2 publisher.
.3 OrderPublisher.java.
.3 OrdersID.java.
.2 SpringbootRabbitmqExampleApplication.java.
} 
\end{flushleft}

\section{Gestione salvataggio ordini}
Per poter permettere il salvataggio persistente degli ordini ho utilizzato il \acrshort{dbms} PostgreSQL. Il database è composto da una singola tabella \texttt{orders} e gli ordini sono distinti univocamente dalla coppia (\texttt{orderId},\texttt{restaurant}).
\begin{table}[H]
    \centering
    \begin{tabular}{|p{3cm}|}
        \hline
        \textbf{orders}\\
        \hline
        \underline{orderId}\\
        \underline{restaurant}\\
        name\\
        qty\\
        price\\
        \hline
    \end{tabular}
    \caption{Tabella \texttt{orders}}
    \label{tab:orders}
\end{table}
La tabella \ref{tab:orders} viene generata dal codice contenuto nel file \texttt{Order}.
\begin{lstlisting}[language=Java, caption=Frammento codice tabella \texttt{orders}, basicstyle=\footnotesize]
@Entity
@Table(name = "orders")
@IdClass(OrdersID.class)
public class Order {
    @Id
    private String orderId;
    @Id
    private String restaurant;
    private String name;
    private int qty;
    private double price;
}
\end{lstlisting}
Nell'ordine, le annotazioni utilizzate servono per
\begin{enumerate}
    \item Indicare questa classe come modello per la generazione della tabella
    \item Specificare il nome della tabella
    \item Utilizzare una \textit{primary key} composta, creando anche una classe che racchiuda gli oggetti che compongono la \textit{primary key} stessa
\end{enumerate}
\begin{lstlisting}[language=Java, caption=Classe per utilizzare \textit{primary key} composta, basicstyle=\footnotesize]
public class OrdersID implements Serializable {
    private String orderId;
    private String restaurant;
}
\end{lstlisting}

\section{Gestione di invio e ricezione degli ordini}

Per poter permettere lo scambio di informazioni nell'applicazione ho configurato il broker di messaggi \textit{RabbitMQ} \ref{rabbitmq} nella classe \texttt{MessagingConfig} specificando i nomi di \textit{queue} ed \textit{exchange}.\\ L'invio degli ordini avviene mediante chiamate REST utilizzando Postman per l'invio della \textit{request} e la visualizzazione della \textit{response}. Una volta inviato l'ordine \texttt{OrderPublisher} ha il compito di ricevere queste chiamate ed estrarre le informazioni utili: nel body (in formato JSON) della chiamata vi è l'ordine mentre nel \textit{path} di quest'ultima vi è il nome del ristorante.
Esempio di chiamata REST:\\
\texttt{http://localhost:9292/order/lanterna}
\begin{lstlisting}[caption=Esempio contenuto del body,basicstyle=\footnotesize]
{
    "orderId": 1,
    "name": "spaghetti",
    "qty": 4,
    "price": 50
}
\end{lstlisting}
Dopo aver elaborato i dati ricevuti la classe \texttt{OrderPublisher} li invia al broker utilizzando l'oggetto \texttt{template} e successivamente li salva in modo persistente richiamando il metodo \texttt{save} dell'oggetto \texttt{orderRepo}:
\begin{lstlisting}[language=Java, caption=Frammento del codice della classe \texttt{OrderPublisher} e relativo metodo \texttt{bookOrder}, basicstyle=\footnotesize]
@Autowired
private RabbitTemplate template;
@Autowired
private OrderRepository orderRepo;
...
template.convertAndSend(MessagingConfig.EXCHANGE,
    MessagingConfig.ROUTING_KEY, orderStatus);
orderRepo.save(order);
...
\end{lstlisting}
Per finire nel progetto è stato inserito anche una classe \texttt{User} il cui compito è quello di ricevere i messaggi dal broker, leggerli e stampare a schermo i dettagli dell'ordinazione appena effettuata.
\begin{lstlisting}[language=Java, caption=Frammento del codice della classe \texttt{User}, basicstyle=\footnotesize]
@RabbitListener(queues = MessagingConfig.QUEUE)
public void consumeMessageFromQueue(OrderStatus orderStatus) {
    System.out.println("Message recieved: " + orderStatus +
    "\nOrder details: " + orderStatus.getOrder().toString());
}
\end{lstlisting}